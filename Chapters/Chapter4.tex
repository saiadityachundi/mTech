% Chapter Template

\chapter{Conclusion} % Main chapter title

\label{Chapter4} % Change X to a consecutive number; for referencing this chapter elsewhere, use \ref{ChapterX}

\lhead{Chapter 4. \emph{Conclusion}} % Change X to a consecutive number; this is for the header on each page - perhaps a shortened title

We proposed and implemented a novel communication architecture for emerging applications in UAVs, using off the shelf hardware. The architecture blends both short range and long range communication systems. For short range communication, we realize a wifi mesh network among the UAVs, using commercially available wifi routers. We achieve this by using an embedded linux operating system for routers called, OpenWrt. This mesh network enables robust and reliable communication among the UAVs for distributed applications like swarm formation and control, localization, cooperative search, to name a few.

We integrate this architecture with a long range communication system for communication with the ground station. We use popular 900Mhz radios called RFD900x as our long range modules. We propose two architectures, \textit{Static Leader} and \textit{Dynamic Leader}, where the second one is an extension of the first one. In \textit{Static Leader}, we designate one of the UAVs as \textit{leader} which transmits data to the ground station. In \textit{Dynamic Leader}, the \textit{leader} is not set apriori, but chosen dynamically by the UAVs themselves.

The integration of the long range communication sytem with the short range mesh network enhances the effective range of the operations that can be conducted with the  UAVs. Furthermore, our architecture is integrated into ROS. This makes our architecture quite flexible, in the sense that it enables anyone to develop their multiple UAV applications in ROS without worrying about the underlying communication architecture. Both the \textit{Static Leader} and \textit{Dynamic Leader} architectures are implemented in two open source ROS packages \textit{serialros} and \textit{swarmBaba}. The source code and implementation can be found at \url{https://github.com/saiadityachundi/serialros} and \url{https://github.com/saiadityachundi/swarmBaba}.

\section{Future Work}
While we have designed and implemented this architecture and tested it indoors, we haven't done extensive outdoor testing. Hence, one of the directions for future work is to do extensive outdoor testing and measure the range over which this architecture reliably works.

Also, the long range radio modules we have used were RFD900x which were serial radios. There are some alternatives to these radios which offer IP based solutions in 900Mhz spectrum. While it would be much easier to integrate an IP based radio into the ROS framework, it is still worthy exercise. There are other proprietary radio solutions in 2.4Ghz spectrum, apart from the normal 802.11 standard, which claim to offer much longer ranges than traditional wifi. Working with these different radios and implementing a similar architecture is worth pursuing.